% -------------------------------------------------------------------------
% ------ nuweb macros (redefine as desired, or omit with "nuweb -p") ------
% -------------------------------------------------------------------------
\providecommand{\NWtxtMacroDefBy}{Macro defined by}
\providecommand{\NWtxtMacroRefIn}{Macro referenced in}
\providecommand{\NWtxtMacroNoRef}{Macro never referenced}
\providecommand{\NWtxtDefBy}{Defined by}
\providecommand{\NWtxtRefIn}{Referenced in}
\providecommand{\NWtxtNoRef}{Not referenced}
\providecommand{\NWtxtFileDefBy}{File defined by}
\providecommand{\NWsep}{${\diamond}$}
\providecommand{\NWlink}[2]{\hyperlink{#1}{#2}}
\providecommand{\NWtarget}[2]{% move baseline up by \baselineskip 
  \raisebox{\baselineskip}[1.5ex][0ex]{%
    \mbox{%
      \hypertarget{#1}{%
        \raisebox{-1\baselineskip}[0ex][0ex]{%
          \mbox{#2}%
}}}}}
% -------------------------------------------------------------------------

\documentclass[11pt,oneside]{article}	%use"amsart"insteadof"article"forAMSLaTeXformat
\usepackage{geometry}		%Seegeometry.pdftolearnthelayoutoptions.Therearelots.
\geometry{letterpaper}		%...ora4paperora5paperor...
%\geometry{landscape}		%Activateforforrotatedpagegeometry
%\usepackage[parfill]{parskip}		%Activatetobeginparagraphswithanemptylineratherthananindent
\usepackage{graphicx}				%Usepdf,png,jpg,orepsßwithpdflatex;useepsinDVImode
								%TeXwillautomaticallyconverteps-->pdfinpdflatex		
\usepackage{amssymb}
\usepackage{hyperref}

%----macros begin---------------------------------------------------------------
\usepackage{color}
\usepackage{amsthm}

\def\conv{\mbox{\textrm{conv}\,}}
\def\aff{\mbox{\textrm{aff}\,}}
\def\E{\mathbb{E}}
\def\R{\mathbb{R}}
\def\Z{\mathbb{Z}}
\def\tex{\TeX}
\def\latex{\LaTeX}
\def\v#1{{\bf #1}}
\def\p#1{{\bf #1}}
\def\T#1{{\bf #1}}

\def\vet#1{{\left(\begin{array}{cccccccccccccccccccc}#1\end{array}\right)}}
\def\mat#1{{\left(\begin{array}{cccccccccccccccccccc}#1\end{array}\right)}}

\def\lin{\mbox{\rm lin}\,}
\def\aff{\mbox{\rm aff}\,}
\def\pos{\mbox{\rm pos}\,}
\def\cone{\mbox{\rm cone}\,}
\def\conv{\mbox{\rm conv}\,}
\newcommand{\homog}[0]{\mbox{\rm homog}\,}
\newcommand{\relint}[0]{\mbox{\rm relint}\,}

%----macros end-----------------------------------------------------------------

\title{Domain mapping with LAR
\footnote{This document is part of the \emph{Linear Algebraic Representation with CoChains} (LAR-CC) framework~\cite{cclar-proj:2013:00}. \today}
}
\author{Alberto Paoluzzi}
%\date{}							%Activatetodisplayagivendateornodate

\begin{document}
\maketitle
\nonstopmode

\begin{abstract}
In this module a first implementation (no optimisations) is done of the \texttt{LARMAP} operator, reproducing the behaviour of the plasm \texttt{MAP} primitive, but with better handling of the topology, including the sewing of decomposed (simplicial domains) about their possible sewing.
\end{abstract}

\tableofcontents

%===============================================================================
\section{Domain decomposition}
%===============================================================================

\paragraph{Standard and scaled decomposition of unit domain}

%-------------------------------------------------------------------------------
\begin{flushleft} \small
\begin{minipage}{\linewidth} \label{scrap1}
\protect\makebox[0ex][r]{\NWtarget{nuweb1}{\rule{0ex}{0ex}}\hspace{1em}}$\langle\,$Generate a simplicial decomposition ot the $[0,1]^d$ domain\nobreak\ {\footnotesize 1}$\,\rangle\equiv$
\vspace{-1ex}
\begin{list}{}{} \item
\mbox{}\verb@def larDomain(shape):@\\
\mbox{}\verb@   V,CV = larSimplexGrid(shape)@\\
\mbox{}\verb@   V = scalePoints(V, [1./d for d in shape])@\\
\mbox{}\verb@   return V,CV@\\
\mbox{}\verb@@\\
\mbox{}\verb@def larIntervals(shape):@\\
\mbox{}\verb@   def larIntervals0(size):@\\
\mbox{}\verb@      V,CV = larDomain(shape)@\\
\mbox{}\verb@      V = scalePoints(V, [scaleFactor for scaleFactor in size])@\\
\mbox{}\verb@      return V,CV@\\
\mbox{}\verb@   return larIntervals0@\\
\mbox{}\verb@@{\NWsep}
\end{list}
\vspace{-1ex}
\footnotesize\addtolength{\baselineskip}{-1ex}
\begin{list}{}{\setlength{\itemsep}{-\parsep}\setlength{\itemindent}{-\leftmargin}}
\item \NWtxtMacroRefIn\ \NWlink{nuweb7b}{7b}.
\end{list}
\end{minipage}\\[4ex]
\end{flushleft}
%-------------------------------------------------------------------------------

%===============================================================================
\section{Embedding via coordinate functions}
%===============================================================================

\subsection{Mapping domain vertices}
%-------------------------------------------------------------------------------
\begin{flushleft} \small \label{scrap2}
\protect\makebox[0ex][r]{\NWtarget{nuweb2a}{\rule{0ex}{0ex}}\hspace{1em}}$\langle\,$Primitive mapping function\nobreak\ {\footnotesize 2a}$\,\rangle\equiv$
\vspace{-1ex}
\begin{list}{}{} \item
\mbox{}\verb@def larMap(coordFuncs):@\\
\mbox{}\verb@   def larMap0(domain):@\\
\mbox{}\verb@      V,CV = domain@\\
\mbox{}\verb@      V = TRANS(CONS(coordFuncs)(V))@\\
\mbox{}\verb@      return V,CV@\\
\mbox{}\verb@   return larMap0@\\
\mbox{}\verb@@{\NWsep}
\end{list}
\vspace{-1ex}
\footnotesize\addtolength{\baselineskip}{-1ex}
\begin{list}{}{\setlength{\itemsep}{-\parsep}\setlength{\itemindent}{-\leftmargin}}
\item \NWtxtMacroRefIn\ \NWlink{nuweb7b}{7b}.
\end{list}
\end{flushleft}
%-------------------------------------------------------------------------------

\subsection{Identify close or coincident points}

%-------------------------------------------------------------------------------
\begin{flushleft} \small \label{scrap3}
\protect\makebox[0ex][r]{\NWtarget{nuweb2b}{\rule{0ex}{0ex}}\hspace{1em}}$\langle\,$Create a dictionary with key the point location\nobreak\ {\footnotesize 2b}$\,\rangle\equiv$
\vspace{-1ex}
\begin{list}{}{} \item
\mbox{}\verb@def checkModel(model):@\\
\mbox{}\verb@   V,CV = model; n = len(V)@\\
\mbox{}\verb@   vertDict = defaultdict(list)@\\
\mbox{}\verb@   for k,v in enumerate(V): vertDict[vcode(v)].append(k) @\\
\mbox{}\verb@   verts = (vertDict.values())@\\
\mbox{}\verb@   invertedindex = [None]*n@\\
\mbox{}\verb@   for k,value in enumerate(verts):@\\
\mbox{}\verb@      print "\n len(value) =",len(value)@\\
\mbox{}\verb@      for i in value:@\\
\mbox{}\verb@         invertedindex[i]=value[0]  @\\
\mbox{}\verb@   CV = [[invertedindex[v] for v in cell] for cell in CV]@\\
\mbox{}\verb@   # filter out degenerate cells@\\
\mbox{}\verb@   CV = [list(set(cell)) for cell in CV if len(set(cell))==len(cell)]@\\
\mbox{}\verb@   return V, CV@\\
\mbox{}\verb@@{\NWsep}
\end{list}
\vspace{-1ex}
\footnotesize\addtolength{\baselineskip}{-1ex}
\begin{list}{}{\setlength{\itemsep}{-\parsep}\setlength{\itemindent}{-\leftmargin}}
\item \NWtxtMacroRefIn\ \NWlink{nuweb7b}{7b}.
\end{list}
\end{flushleft}
%-------------------------------------------------------------------------------


%===============================================================================
\section{Primitive objets}
%===============================================================================
%-------------------------------------------------------------------------------
\subsection{1D primitives}
%-------------------------------------------------------------------------------

\paragraph{Curved line}
%-------------------------------------------------------------------------------
\begin{flushleft} \small \label{scrap4}
\protect\makebox[0ex][r]{\NWtarget{nuweb2c}{\rule{0ex}{0ex}}\hspace{1em}}$\langle\,$aaaa\nobreak\ {\footnotesize 2c}$\,\rangle\equiv$
\vspace{-1ex}
\begin{list}{}{} \item
\mbox{}\verb@@\\
\mbox{}\verb@@\\
\mbox{}\verb@@{\NWsep}
\end{list}
\vspace{-1ex}
\footnotesize\addtolength{\baselineskip}{-1ex}
\begin{list}{}{\setlength{\itemsep}{-\parsep}\setlength{\itemindent}{-\leftmargin}}
\item {\NWtxtMacroNoRef}.
\end{list}
\end{flushleft}
%-------------------------------------------------------------------------------

\paragraph{Circle}
%-------------------------------------------------------------------------------
\begin{flushleft} \small \label{scrap5}
\protect\makebox[0ex][r]{\NWtarget{nuweb3a}{\rule{0ex}{0ex}}\hspace{1em}}$\langle\,$Circle centered in the origin\nobreak\ {\footnotesize 3a}$\,\rangle\equiv$
\vspace{-1ex}
\begin{list}{}{} \item
\mbox{}\verb@def larCircle(radius=1.):@\\
\mbox{}\verb@   def larCircle0(shape=36):@\\
\mbox{}\verb@      domain = larIntervals([shape])([2*PI])@\\
\mbox{}\verb@      V,CV = domain@\\
\mbox{}\verb@      x = lambda coords : [radius*COS(p[0]) for p in V]@\\
\mbox{}\verb@      y = lambda coords : [radius*SIN(p[0]) for p in V]@\\
\mbox{}\verb@      mapping = [x,y]@\\
\mbox{}\verb@      return larMap(mapping)(domain)@\\
\mbox{}\verb@   return larCircle0@\\
\mbox{}\verb@@{\NWsep}
\end{list}
\vspace{-1ex}
\footnotesize\addtolength{\baselineskip}{-1ex}
\begin{list}{}{\setlength{\itemsep}{-\parsep}\setlength{\itemindent}{-\leftmargin}}
\item \NWtxtMacroRefIn\ \NWlink{nuweb7b}{7b}.
\end{list}
\end{flushleft}
%-------------------------------------------------------------------------------
%-------------------------------------------------------------------------------
\subsection{2D primitives}
%-------------------------------------------------------------------------------

\paragraph{Disk}
-------------------------------------------------------------------------------
\begin{flushleft} \small \label{scrap6}
\protect\makebox[0ex][r]{\NWtarget{nuweb3b}{\rule{0ex}{0ex}}\hspace{1em}}$\langle\,$Disk centered in the origin\nobreak\ {\footnotesize 3b}$\,\rangle\equiv$
\vspace{-1ex}
\begin{list}{}{} \item
\mbox{}\verb@def larDisk(radius=1.):@\\
\mbox{}\verb@   def larDisk0(shape=[36,1]):@\\
\mbox{}\verb@      domain = larIntervals(shape)([2*PI,radius])@\\
\mbox{}\verb@      V,CV = domain@\\
\mbox{}\verb@      x = lambda V : [p[1]*COS(p[0]) for p in V]@\\
\mbox{}\verb@      y = lambda V : [p[1]*SIN(p[0]) for p in V]@\\
\mbox{}\verb@      mapping = [x,y]@\\
\mbox{}\verb@      return larMap(mapping)(domain)@\\
\mbox{}\verb@   return larDisk0@\\
\mbox{}\verb@@\\
\mbox{}\verb@@{\NWsep}
\end{list}
\vspace{-1ex}
\footnotesize\addtolength{\baselineskip}{-1ex}
\begin{list}{}{\setlength{\itemsep}{-\parsep}\setlength{\itemindent}{-\leftmargin}}
\item \NWtxtMacroRefIn\ \NWlink{nuweb7b}{7b}.
\end{list}
\end{flushleft}
%-------------------------------------------------------------------------------

\paragraph{Ring}
-------------------------------------------------------------------------------
\begin{flushleft} \small \label{scrap7}
\protect\makebox[0ex][r]{\NWtarget{nuweb3c}{\rule{0ex}{0ex}}\hspace{1em}}$\langle\,$Ring centered in the origin\nobreak\ {\footnotesize 3c}$\,\rangle\equiv$
\vspace{-1ex}
\begin{list}{}{} \item
\mbox{}\verb@def larRing(params):@\\
\mbox{}\verb@   r1,r2 = params@\\
\mbox{}\verb@   def larDisk0(shape=[36,1]):@\\
\mbox{}\verb@      V,CV = larIntervals(shape)([2*PI,r2-r1])@\\
\mbox{}\verb@      V = translatePoints(V,[0,r1])@\\
\mbox{}\verb@      domain = V,CV@\\
\mbox{}\verb@      VIEW(EXPLODE(1.2,1.2,1.2)(MKPOLS((V,CV))))@\\
\mbox{}\verb@      x = lambda V : [p[1] * COS(p[0]) for p in V]@\\
\mbox{}\verb@      y = lambda V : [p[1] * SIN(p[0]) for p in V]@\\
\mbox{}\verb@      mapping = [x,y]@\\
\mbox{}\verb@      return larMap(mapping)(domain)@\\
\mbox{}\verb@   return larDisk0@\\
\mbox{}\verb@@\\
\mbox{}\verb@@{\NWsep}
\end{list}
\vspace{-1ex}
\footnotesize\addtolength{\baselineskip}{-1ex}
\begin{list}{}{\setlength{\itemsep}{-\parsep}\setlength{\itemindent}{-\leftmargin}}
\item \NWtxtMacroRefIn\ \NWlink{nuweb7b}{7b}.
\end{list}
\end{flushleft}
%-------------------------------------------------------------------------------
\paragraph{Cylinder surface}
%-------------------------------------------------------------------------------
\begin{flushleft} \small \label{scrap8}
\protect\makebox[0ex][r]{\NWtarget{nuweb4a}{\rule{0ex}{0ex}}\hspace{1em}}$\langle\,$Cylinder surface with $z$ axis\nobreak\ {\footnotesize 4a}$\,\rangle\equiv$
\vspace{-1ex}
\begin{list}{}{} \item
\mbox{}\verb@from scipy.linalg import det@\\
\mbox{}\verb@"""@\\
\mbox{}\verb@def makeOriented(model):@\\
\mbox{}\verb@   V,CV = model@\\
\mbox{}\verb@   out = []@\\
\mbox{}\verb@   for cell in CV: @\\
\mbox{}\verb@      mat = scipy.array([V[v]+[1] for v in cell]+[[0,0,0,1]])@\\
\mbox{}\verb@      if det(mat) < 0.0:@\\
\mbox{}\verb@         out.append(cell)@\\
\mbox{}\verb@      else:@\\
\mbox{}\verb@         out.append([cell[1]]+[cell[0]]+cell[2:])@\\
\mbox{}\verb@      print "\n det(mat) =",det(mat)@\\
\mbox{}\verb@   return V,out@\\
\mbox{}\verb@"""@\\
\mbox{}\verb@def larCylinder(params):@\\
\mbox{}\verb@   radius,height= params@\\
\mbox{}\verb@   def larCylinder0(shape=[36,1]):@\\
\mbox{}\verb@      domain = larIntervals(shape)([2*PI,1])@\\
\mbox{}\verb@      V,CV = domain@\\
\mbox{}\verb@      x = lambda V : [radius*COS(p[0]) for p in V]@\\
\mbox{}\verb@      y = lambda V : [radius*SIN(p[0]) for p in V]@\\
\mbox{}\verb@      z = lambda V : [height*p[1] for p in V]@\\
\mbox{}\verb@      mapping = [x,y,z]@\\
\mbox{}\verb@      model = larMap(mapping)(domain)@\\
\mbox{}\verb@      # model = makeOriented(model)@\\
\mbox{}\verb@      return model@\\
\mbox{}\verb@   return larCylinder0@\\
\mbox{}\verb@@{\NWsep}
\end{list}
\vspace{-1ex}
\footnotesize\addtolength{\baselineskip}{-1ex}
\begin{list}{}{\setlength{\itemsep}{-\parsep}\setlength{\itemindent}{-\leftmargin}}
\item \NWtxtMacroRefIn\ \NWlink{nuweb7b}{7b}.
\end{list}
\end{flushleft}
%-------------------------------------------------------------------------------
\paragraph{Spherical surface of given radius}
%-------------------------------------------------------------------------------
\begin{flushleft} \small \label{scrap9}
\protect\makebox[0ex][r]{\NWtarget{nuweb4b}{\rule{0ex}{0ex}}\hspace{1em}}$\langle\,$Spherical surface of given radius\nobreak\ {\footnotesize 4b}$\,\rangle\equiv$
\vspace{-1ex}
\begin{list}{}{} \item
\mbox{}\verb@def larSphere(radius=1):@\\
\mbox{}\verb@   def larSphere0(shape=[18,36]):@\\
\mbox{}\verb@      V,CV = larIntervals(shape)([PI,2*PI])@\\
\mbox{}\verb@      V = translatePoints(V,[-PI/2,-PI])@\\
\mbox{}\verb@      domain = V,CV@\\
\mbox{}\verb@      x = lambda V : [radius*COS(p[0])*SIN(p[1]) for p in V]@\\
\mbox{}\verb@      y = lambda V : [radius*COS(p[0])*COS(p[1]) for p in V]@\\
\mbox{}\verb@      z = lambda V : [radius*SIN(p[0]) for p in V]@\\
\mbox{}\verb@      mapping = [x,y,z]@\\
\mbox{}\verb@      return larMap(mapping)(domain)@\\
\mbox{}\verb@   return larSphere0@\\
\mbox{}\verb@@{\NWsep}
\end{list}
\vspace{-1ex}
\footnotesize\addtolength{\baselineskip}{-1ex}
\begin{list}{}{\setlength{\itemsep}{-\parsep}\setlength{\itemindent}{-\leftmargin}}
\item \NWtxtMacroRefIn\ \NWlink{nuweb7b}{7b}.
\end{list}
\end{flushleft}
%-------------------------------------------------------------------------------
\paragraph{Toroidal surface}
%-------------------------------------------------------------------------------
\begin{flushleft} \small \label{scrap10}
\protect\makebox[0ex][r]{\NWtarget{nuweb5a}{\rule{0ex}{0ex}}\hspace{1em}}$\langle\,$Toroidal surface of given radiuses\nobreak\ {\footnotesize 5a}$\,\rangle\equiv$
\vspace{-1ex}
\begin{list}{}{} \item
\mbox{}\verb@def larToroidal(params):@\\
\mbox{}\verb@   r,R = params@\\
\mbox{}\verb@   def larToroidal0(shape=[24,36]):@\\
\mbox{}\verb@      domain = larIntervals(shape)([2*PI,2*PI])@\\
\mbox{}\verb@      V,CV = domain@\\
\mbox{}\verb@      x = lambda V : [(R + r*COS(p[0])) * COS(p[1]) for p in V]@\\
\mbox{}\verb@      y = lambda V : [(R + r*COS(p[0])) * SIN(p[1]) for p in V]@\\
\mbox{}\verb@      z = lambda V : [-r * SIN(p[0]) for p in V]@\\
\mbox{}\verb@      mapping = [x,y,z]@\\
\mbox{}\verb@      return larMap(mapping)(domain)@\\
\mbox{}\verb@   return larToroidal0@\\
\mbox{}\verb@@{\NWsep}
\end{list}
\vspace{-1ex}
\footnotesize\addtolength{\baselineskip}{-1ex}
\begin{list}{}{\setlength{\itemsep}{-\parsep}\setlength{\itemindent}{-\leftmargin}}
\item \NWtxtMacroRefIn\ \NWlink{nuweb7b}{7b}.
\end{list}
\end{flushleft}
%-------------------------------------------------------------------------------
\paragraph{Crown surface}
%-------------------------------------------------------------------------------
\begin{flushleft} \small \label{scrap11}
\protect\makebox[0ex][r]{\NWtarget{nuweb5b}{\rule{0ex}{0ex}}\hspace{1em}}$\langle\,$Half-toroidal surface of given radiuses\nobreak\ {\footnotesize 5b}$\,\rangle\equiv$
\vspace{-1ex}
\begin{list}{}{} \item
\mbox{}\verb@def larCrown(params):@\\
\mbox{}\verb@   r,R = params@\\
\mbox{}\verb@   def larCrown0(shape=[24,36]):@\\
\mbox{}\verb@      V,CV = larIntervals(shape)([PI,2*PI])@\\
\mbox{}\verb@      V = translatePoints(V,[-PI/2,0])@\\
\mbox{}\verb@      domain = V,CV@\\
\mbox{}\verb@      x = lambda V : [(R + r*COS(p[0])) * COS(p[1]) for p in V]@\\
\mbox{}\verb@      y = lambda V : [(R + r*COS(p[0])) * SIN(p[1]) for p in V]@\\
\mbox{}\verb@      z = lambda V : [-r * SIN(p[0]) for p in V]@\\
\mbox{}\verb@      mapping = [x,y,z]@\\
\mbox{}\verb@      return larMap(mapping)(domain)@\\
\mbox{}\verb@   return larCrown0@\\
\mbox{}\verb@@{\NWsep}
\end{list}
\vspace{-1ex}
\footnotesize\addtolength{\baselineskip}{-1ex}
\begin{list}{}{\setlength{\itemsep}{-\parsep}\setlength{\itemindent}{-\leftmargin}}
\item \NWtxtMacroRefIn\ \NWlink{nuweb7b}{7b}.
\end{list}
\end{flushleft}
%-------------------------------------------------------------------------------

%-------------------------------------------------------------------------------
\subsection{3D primitives}
%-------------------------------------------------------------------------------

\paragraph{Ball}
%-------------------------------------------------------------------------------
\begin{flushleft} \small \label{scrap12}
\protect\makebox[0ex][r]{\NWtarget{nuweb6a}{\rule{0ex}{0ex}}\hspace{1em}}$\langle\,$Solid Sphere of given radius\nobreak\ {\footnotesize 6a}$\,\rangle\equiv$
\vspace{-1ex}
\begin{list}{}{} \item
\mbox{}\verb@def larBall(radius=1):@\\
\mbox{}\verb@   def larBall0(shape=[18,36]):@\\
\mbox{}\verb@      V,CV = checkModel(larSphere(radius)(shape))@\\
\mbox{}\verb@      VIEW(STRUCT(MKPOLS((V,CV))))@\\
\mbox{}\verb@      return V,[range(len(V))]@\\
\mbox{}\verb@   return larBall0@\\
\mbox{}\verb@@{\NWsep}
\end{list}
\vspace{-1ex}
\footnotesize\addtolength{\baselineskip}{-1ex}
\begin{list}{}{\setlength{\itemsep}{-\parsep}\setlength{\itemindent}{-\leftmargin}}
\item \NWtxtMacroRefIn\ \NWlink{nuweb7b}{7b}.
\end{list}
\end{flushleft}
%-------------------------------------------------------------------------------

\paragraph{Solid cylinder}
%-------------------------------------------------------------------------------
\begin{flushleft} \small \label{scrap13}
\protect\makebox[0ex][r]{\NWtarget{nuweb6b}{\rule{0ex}{0ex}}\hspace{1em}}$\langle\,$Solid cylinder of given radius and height\nobreak\ {\footnotesize 6b}$\,\rangle\equiv$
\vspace{-1ex}
\begin{list}{}{} \item
\mbox{}\verb@def larRod(params):@\\
\mbox{}\verb@   radius,height= params@\\
\mbox{}\verb@   def larRod0(shape=[36,1]):@\\
\mbox{}\verb@      V,CV = checkModel(larCylinder(params)(shape))@\\
\mbox{}\verb@      VIEW(STRUCT(MKPOLS((V,CV))))@\\
\mbox{}\verb@      return V,[range(len(V))]@\\
\mbox{}\verb@   return larRod0@\\
\mbox{}\verb@@{\NWsep}
\end{list}
\vspace{-1ex}
\footnotesize\addtolength{\baselineskip}{-1ex}
\begin{list}{}{\setlength{\itemsep}{-\parsep}\setlength{\itemindent}{-\leftmargin}}
\item \NWtxtMacroRefIn\ \NWlink{nuweb7b}{7b}.
\end{list}
\end{flushleft}
%-------------------------------------------------------------------------------

\paragraph{Solid torus}
%-------------------------------------------------------------------------------
\begin{flushleft} \small \label{scrap14}
\protect\makebox[0ex][r]{\NWtarget{nuweb6c}{\rule{0ex}{0ex}}\hspace{1em}}$\langle\,$Solid torus of given radiuses\nobreak\ {\footnotesize 6c}$\,\rangle\equiv$
\vspace{-1ex}
\begin{list}{}{} \item
\mbox{}\verb@def larTorus(params):@\\
\mbox{}\verb@   r,R = params@\\
\mbox{}\verb@   def larTorus0(shape=[24,36,1]):@\\
\mbox{}\verb@      domain = larIntervals(shape)([2*PI,2*PI,r])@\\
\mbox{}\verb@      V,CV = domain@\\
\mbox{}\verb@      x = lambda V : [(R + p[2]*COS(p[0])) * COS(p[1]) for p in V]@\\
\mbox{}\verb@      y = lambda V : [(R + p[2]*COS(p[0])) * SIN(p[1]) for p in V]@\\
\mbox{}\verb@      z = lambda V : [-p[2] * SIN(p[0]) for p in V]@\\
\mbox{}\verb@      mapping = [x,y,z]@\\
\mbox{}\verb@      return larMap(mapping)(domain)@\\
\mbox{}\verb@   return larTorus0@\\
\mbox{}\verb@@{\NWsep}
\end{list}
\vspace{-1ex}
\footnotesize\addtolength{\baselineskip}{-1ex}
\begin{list}{}{\setlength{\itemsep}{-\parsep}\setlength{\itemindent}{-\leftmargin}}
\item \NWtxtMacroRefIn\ \NWlink{nuweb7b}{7b}.
\end{list}
\end{flushleft}
%-------------------------------------------------------------------------------

\paragraph{Solid pizza}
%-------------------------------------------------------------------------------
\begin{flushleft} \small \label{scrap15}
\protect\makebox[0ex][r]{\NWtarget{nuweb7a}{\rule{0ex}{0ex}}\hspace{1em}}$\langle\,$Solid pizza of given radiuses\nobreak\ {\footnotesize 7a}$\,\rangle\equiv$
\vspace{-1ex}
\begin{list}{}{} \item
\mbox{}\verb@def larPizza(params):@\\
\mbox{}\verb@   r,R= params@\\
\mbox{}\verb@   def larPizza0(shape=[24,36]):@\\
\mbox{}\verb@      V,CV = checkModel(larCrown(params)(shape))@\\
\mbox{}\verb@      VIEW(STRUCT(MKPOLS((V,CV))))@\\
\mbox{}\verb@      return V,[range(len(V))]@\\
\mbox{}\verb@   return larPizza0@\\
\mbox{}\verb@@{\NWsep}
\end{list}
\vspace{-1ex}
\footnotesize\addtolength{\baselineskip}{-1ex}
\begin{list}{}{\setlength{\itemsep}{-\parsep}\setlength{\itemindent}{-\leftmargin}}
\item \NWtxtMacroRefIn\ \NWlink{nuweb7b}{7b}.
\end{list}
\end{flushleft}
%-------------------------------------------------------------------------------

%===============================================================================
\section{Exporting the library}
%===============================================================================
%-------------------------------------------------------------------------------
\begin{flushleft} \small \label{scrap16}
\protect\makebox[0ex][r]{\NWtarget{nuweb7b}{\rule{0ex}{0ex}}\hspace{1em}}\verb@"lib/py/mapper.py"@\nobreak\ {\footnotesize 7b }$\equiv$
\vspace{-1ex}
\begin{list}{}{} \item
\mbox{}\verb@""" Mapping functions and primitive objects """@\\
\mbox{}\verb@@\hbox{$\langle\,$Initial import of modules\nobreak\ {\footnotesize \NWlink{nuweb9a}{9a}}$\,\rangle$}\verb@@\\
\mbox{}\verb@@\hbox{$\langle\,$Generate a simplicial decomposition ot the $[0,1]^d$ domain\nobreak\ {\footnotesize \NWlink{nuweb1}{1}}$\,\rangle$}\verb@@\\
\mbox{}\verb@@\hbox{$\langle\,$Create a dictionary with key the point location\nobreak\ {\footnotesize \NWlink{nuweb2b}{2b}}$\,\rangle$}\verb@@\\
\mbox{}\verb@@\hbox{$\langle\,$Primitive mapping function\nobreak\ {\footnotesize \NWlink{nuweb2a}{2a}}$\,\rangle$}\verb@@\\
\mbox{}\verb@@\hbox{$\langle\,$Basic tests of mapper module\nobreak\ {\footnotesize \NWlink{nuweb7c}{7c}}$\,\rangle$}\verb@@\\
\mbox{}\verb@@\hbox{$\langle\,$Circle centered in the origin\nobreak\ {\footnotesize \NWlink{nuweb3a}{3a}}$\,\rangle$}\verb@@\\
\mbox{}\verb@@\hbox{$\langle\,$Disk centered in the origin\nobreak\ {\footnotesize \NWlink{nuweb3b}{3b}}$\,\rangle$}\verb@@\\
\mbox{}\verb@@\hbox{$\langle\,$Ring centered in the origin\nobreak\ {\footnotesize \NWlink{nuweb3c}{3c}}$\,\rangle$}\verb@@\\
\mbox{}\verb@@\hbox{$\langle\,$Spherical surface of given radius\nobreak\ {\footnotesize \NWlink{nuweb4b}{4b}}$\,\rangle$}\verb@@\\
\mbox{}\verb@@\hbox{$\langle\,$Cylinder surface with $z$ axis\nobreak\ {\footnotesize \NWlink{nuweb4a}{4a}}$\,\rangle$}\verb@@\\
\mbox{}\verb@@\hbox{$\langle\,$Toroidal surface of given radiuses\nobreak\ {\footnotesize \NWlink{nuweb5a}{5a}}$\,\rangle$}\verb@@\\
\mbox{}\verb@@\hbox{$\langle\,$Half-toroidal surface of given radiuses\nobreak\ {\footnotesize \NWlink{nuweb5b}{5b}}$\,\rangle$}\verb@@\\
\mbox{}\verb@@\hbox{$\langle\,$Solid Sphere of given radius\nobreak\ {\footnotesize \NWlink{nuweb6a}{6a}}$\,\rangle$}\verb@@\\
\mbox{}\verb@@\hbox{$\langle\,$Solid cylinder of given radius and height\nobreak\ {\footnotesize \NWlink{nuweb6b}{6b}}$\,\rangle$}\verb@@\\
\mbox{}\verb@@\hbox{$\langle\,$Solid torus of given radiuses\nobreak\ {\footnotesize \NWlink{nuweb6c}{6c}}$\,\rangle$}\verb@@\\
\mbox{}\verb@@\hbox{$\langle\,$Solid pizza of given radiuses\nobreak\ {\footnotesize \NWlink{nuweb7a}{7a}}$\,\rangle$}\verb@@\\
\mbox{}\verb@@{\NWsep}
\end{list}
\vspace{-2ex}
\end{flushleft}
%-------------------------------------------------------------------------------
%===============================================================================
\section{Examples}
%===============================================================================
%===============================================================================
\section{Tests}
%===============================================================================

	
%-------------------------------------------------------------------------------
\begin{flushleft} \small \label{scrap17}
\protect\makebox[0ex][r]{\NWtarget{nuweb7c}{\rule{0ex}{0ex}}\hspace{1em}}$\langle\,$Basic tests of mapper module\nobreak\ {\footnotesize 7c}$\,\rangle\equiv$
\vspace{-1ex}
\begin{list}{}{} \item
\mbox{}\verb@if __name__=="__main__":@\\
\mbox{}\verb@   V,EV = larDomain([5])@\\
\mbox{}\verb@   VIEW(EXPLODE(1.5,1.5,1.5)(MKPOLS((V,EV))))@\\
\mbox{}\verb@   V,EV = larIntervals([24])([2*PI])@\\
\mbox{}\verb@   VIEW(EXPLODE(1.5,1.5,1.5)(MKPOLS((V,EV))))@\\
\mbox{}\verb@      @\\
\mbox{}\verb@   V,FV = larDomain([5,3])@\\
\mbox{}\verb@   VIEW(EXPLODE(1.5,1.5,1.5)(MKPOLS((V,FV))))@\\
\mbox{}\verb@   V,FV = larIntervals([36,3])([2*PI,1.])@\\
\mbox{}\verb@   VIEW(EXPLODE(1.5,1.5,1.5)(MKPOLS((V,FV))))@\\
\mbox{}\verb@      @\\
\mbox{}\verb@   V,CV = larDomain([5,3,1])@\\
\mbox{}\verb@   VIEW(EXPLODE(1.5,1.5,1.5)(MKPOLS((V,CV))))@\\
\mbox{}\verb@   V,CV = larIntervals([36,2,3])([2*PI,1.,1.])@\\
\mbox{}\verb@   VIEW(EXPLODE(1.5,1.5,1.5)(MKPOLS((V,CV))))@\\
\mbox{}\verb@@{\NWsep}
\end{list}
\vspace{-1ex}
\footnotesize\addtolength{\baselineskip}{-1ex}
\begin{list}{}{\setlength{\itemsep}{-\parsep}\setlength{\itemindent}{-\leftmargin}}
\item \NWtxtMacroRefIn\ \NWlink{nuweb7b}{7b}.
\end{list}
\end{flushleft}
%-------------------------------------------------------------------------------

\paragraph{Circumference of unit radius}

%-------------------------------------------------------------------------------
\begin{flushleft} \small \label{scrap18}
\protect\makebox[0ex][r]{\NWtarget{nuweb8}{\rule{0ex}{0ex}}\hspace{1em}}\verb@"test/py/mapper/test01.py"@\nobreak\ {\footnotesize 8 }$\equiv$
\vspace{-1ex}
\begin{list}{}{} \item
\mbox{}\verb@""" Circumference of unit radius """@\\
\mbox{}\verb@@\hbox{$\langle\,$Initial import of modules\nobreak\ {\footnotesize \NWlink{nuweb9a}{9a}}$\,\rangle$}\verb@@\\
\mbox{}\verb@from mapper import *@\\
\mbox{}\verb@"""@\\
\mbox{}\verb@model = checkModel(larCircle(1)())@\\
\mbox{}\verb@VIEW(EXPLODE(1.2,1.2,1.2)(MKPOLS(model)))@\\
\mbox{}\verb@model = checkModel(larDisk(1)([36,4]))@\\
\mbox{}\verb@VIEW(EXPLODE(1.2,1.2,1.2)(MKPOLS(model)))@\\
\mbox{}\verb@model = checkModel(larRing([.9, 1.])([36,2]))@\\
\mbox{}\verb@VIEW(EXPLODE(1.2,1.2,1.2)(MKPOLS(model)))@\\
\mbox{}\verb@model = checkModel(larCylinder([.5,2.])([32,1]))@\\
\mbox{}\verb@VIEW(STRUCT(MKPOLS(model)))@\\
\mbox{}\verb@model = checkModel(larSphere(1)())@\\
\mbox{}\verb@VIEW(STRUCT(MKPOLS(model)))@\\
\mbox{}\verb@model = larBall(1)()@\\
\mbox{}\verb@VIEW(STRUCT(MKPOLS(model)))@\\
\mbox{}\verb@model = larRod([.25,2.])([32,1])@\\
\mbox{}\verb@VIEW(STRUCT(MKPOLS(model)))@\\
\mbox{}\verb@model = checkModel(larToroidal([0.5,1])())@\\
\mbox{}\verb@VIEW(STRUCT(MKPOLS(model)))@\\
\mbox{}\verb@model = checkModel(larCrown([0.125,1])([8,48]))@\\
\mbox{}\verb@VIEW(STRUCT(MKPOLS(model)))@\\
\mbox{}\verb@model = larPizza([0.05,1])([8,48])@\\
\mbox{}\verb@VIEW(STRUCT(MKPOLS(model)))@\\
\mbox{}\verb@"""@\\
\mbox{}\verb@model = checkModel(larTorus([0.5,1])())@\\
\mbox{}\verb@VIEW(STRUCT(MKPOLS(model)))@\\
\mbox{}\verb@@{\NWsep}
\end{list}
\vspace{-2ex}
\end{flushleft}
%-------------------------------------------------------------------------------


%===============================================================================
\appendix
\section{Utility functions}
%===============================================================================

%-------------------------------------------------------------------------------
\begin{flushleft} \small \label{scrap19}
\protect\makebox[0ex][r]{\NWtarget{nuweb9a}{\rule{0ex}{0ex}}\hspace{1em}}$\langle\,$Initial import of modules\nobreak\ {\footnotesize 9a}$\,\rangle\equiv$
\vspace{-1ex}
\begin{list}{}{} \item
\mbox{}\verb@from pyplasm import *@\\
\mbox{}\verb@from scipy import *@\\
\mbox{}\verb@import os,sys@\\
\mbox{}\verb@@\\
\mbox{}\verb@""" import modules from larcc/lib """@\\
\mbox{}\verb@sys.path.insert(0, 'lib/py/')@\\
\mbox{}\verb@@\hbox{$\langle\,$Import the module\nobreak\ ({\footnotesize \NWtarget{nuweb9b}{9b}\label{scrap20}
 }\mbox{}\verb@lar2psm@ ) {\footnotesize \NWlink{nuweb9g}{9g}}$\,\rangle$}\verb@@\\
\mbox{}\verb@@\hbox{$\langle\,$Import the module\nobreak\ ({\footnotesize \NWtarget{nuweb9c}{9c}\label{scrap21}
 }\mbox{}\verb@simplexn@ ) {\footnotesize \NWlink{nuweb9g}{9g}}$\,\rangle$}\verb@@\\
\mbox{}\verb@@\hbox{$\langle\,$Import the module\nobreak\ ({\footnotesize \NWtarget{nuweb9d}{9d}\label{scrap22}
 }\mbox{}\verb@larcc@ ) {\footnotesize \NWlink{nuweb9g}{9g}}$\,\rangle$}\verb@@\\
\mbox{}\verb@@\hbox{$\langle\,$Import the module\nobreak\ ({\footnotesize \NWtarget{nuweb9e}{9e}\label{scrap23}
 }\mbox{}\verb@largrid@ ) {\footnotesize \NWlink{nuweb9g}{9g}}$\,\rangle$}\verb@@\\
\mbox{}\verb@@\hbox{$\langle\,$Import the module\nobreak\ ({\footnotesize \NWtarget{nuweb9f}{9f}\label{scrap24}
 }\mbox{}\verb@boolean2@ ) {\footnotesize \NWlink{nuweb9g}{9g}}$\,\rangle$}\verb@@\\
\mbox{}\verb@@{\NWsep}
\end{list}
\vspace{-1ex}
\footnotesize\addtolength{\baselineskip}{-1ex}
\begin{list}{}{\setlength{\itemsep}{-\parsep}\setlength{\itemindent}{-\leftmargin}}
\item \NWtxtMacroRefIn\ \NWlink{nuweb7b}{7b}\NWlink{nuweb8}{, 8}.
\end{list}
\end{flushleft}
%-------------------------------------------------------------------------------

%-------------------------------------------------------------------------------
\begin{flushleft} \small \label{scrap25}
\protect\makebox[0ex][r]{\NWtarget{nuweb9g}{\rule{0ex}{0ex}}\hspace{1em}}$\langle\,$Import the module\nobreak\ {\footnotesize 9g}$\,\rangle\equiv$
\vspace{-1ex}
\begin{list}{}{} \item
\mbox{}\verb@import @@1\verb@@\\
\mbox{}\verb@from @@1\verb@ import *@\\
\mbox{}\verb@@{\NWsep}
\end{list}
\vspace{-1ex}
\footnotesize\addtolength{\baselineskip}{-1ex}
\begin{list}{}{\setlength{\itemsep}{-\parsep}\setlength{\itemindent}{-\leftmargin}}
\item \NWtxtMacroRefIn\ \NWlink{nuweb9a}{9a}.
\end{list}
\end{flushleft}
%-------------------------------------------------------------------------------


\subsection{Numeric utilities}

A small set of utilityy functions is used to transform a point representation as array of coordinates into a string of fixed format to be used as point key into python dictionaries.

%------------------------------------------------------------------
\begin{flushleft} \small \label{scrap26}
\protect\makebox[0ex][r]{\NWtarget{nuweb9h}{\rule{0ex}{0ex}}\hspace{1em}}$\langle\,$Symbolic utility to represent points as strings\nobreak\ {\footnotesize 9h}$\,\rangle\equiv$
\vspace{-1ex}
\begin{list}{}{} \item
\mbox{}\verb@""" TODO: use package Decimal (http://docs.python.org/2/library/decimal.html) """@\\
\mbox{}\verb@ROUND_ZERO = 1E-07@\\
\mbox{}\verb@def round_or_zero (x,prec=7):@\\
\mbox{}\verb@   """@\\
\mbox{}\verb@   Decision procedure to approximate a small number to zero.@\\
\mbox{}\verb@   Return either the input number or zero.@\\
\mbox{}\verb@   """@\\
\mbox{}\verb@   def myround(x):@\\
\mbox{}\verb@      return eval(('%.'+str(prec)+'f') % round(x,prec))@\\
\mbox{}\verb@   xx = myround(x)@\\
\mbox{}\verb@   if abs(xx) < ROUND_ZERO: return 0.0@\\
\mbox{}\verb@   else: return xx@\\
\mbox{}\verb@@\\
\mbox{}\verb@def prepKey (args): return "["+", ".join(args)+"]"@\\
\mbox{}\verb@@\\
\mbox{}\verb@def fixedPrec(value):@\\
\mbox{}\verb@   if abs(value - int(value))<ROUND_ZERO: value = int(value)@\\
\mbox{}\verb@   out = ('%0.7f'% value).rstrip('0')@\\
\mbox{}\verb@   if out == '-0.': out = '0.'@\\
\mbox{}\verb@   return out@\\
\mbox{}\verb@   @\\
\mbox{}\verb@def vcode (vect): @\\
\mbox{}\verb@   """@\\
\mbox{}\verb@   To generate a string representation of a number array.@\\
\mbox{}\verb@   Used to generate the vertex keys in PointSet dictionary, and other similar operations.@\\
\mbox{}\verb@   """@\\
\mbox{}\verb@   return prepKey(AA(fixedPrec)(vect))@\\
\mbox{}\verb@@{\NWsep}
\end{list}
\vspace{-1ex}
\footnotesize\addtolength{\baselineskip}{-1ex}
\begin{list}{}{\setlength{\itemsep}{-\parsep}\setlength{\itemindent}{-\leftmargin}}
\item {\NWtxtMacroNoRef}.
\end{list}
\end{flushleft}
%------------------------------------------------------------------


\bibliographystyle{amsalpha}
\bibliography{mapper}

\end{document}
